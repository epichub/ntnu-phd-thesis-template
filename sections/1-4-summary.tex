\chapter{Research results}
\label{chap:summary}

This chapter describes the results from the PhD project using the
research questions set down in Section~\ref{intro:sec:rquestions}...
\lipsum[1-2]

\section{Research contributions}
\label{sec:contributions}
The research done throughout this thesis has six main contributions.
\lipsum[1-2]
\begin{description}[labelwidth=20pt, leftmargin=!]
\item[C1] contribution 1
\item[C2] contribution 2
\end{description}


\section{List of publications}
\label{summary:publist}
This section gives an overview of the publications included in this thesis. \lipsum[1-2]

The first paper addresses ...
\lipsum[1-2]
\begin{tcolorbox}[title=\cref{pap:paper1} (Mathisen 2016)]
  \label{papersum1}
  \textbf{Title:}\\
  \hspace*{6mm}Data driven case base construction for prediction of success
  \hspace*{6mm}of marine operations
  
  \textbf{Published at conference:}\\
    \hspace*{6mm}Case-Based Reasoning and Deep Learning Workshop -
    \hspace*{6mm}CBRDL-2017
\end{tcolorbox}
%\textbf{Kansje kutte denne delen??}\newline
Following up on the work in \cref{pap:paper1}, bla bl
\lipsum[1-2]

\begin{tcolorbox}[title=\cref{pap:paper2} (Mathisen 2019)]
  \label{papersum2}
  \textbf{Title:}\\
  \hspace*{6mm}Learning similarity measures from data

  \textbf{Publised in journal:}\\
  \hspace*{6mm}Progress in Artificial Intelligence
\end{tcolorbox}
\section{Contributions towards research questions}
\label{results:sec:rqs}

\lipsum[6-7]


\textbf{RQ1: RQ1 text..}
\label{results:sec:rq2}
\lipsum[6-7]

\textbf{RQ2: RQ2 text..}
\label{results:sec:rq3}
\lipsum[6-7]

\section{Summary of auxiliary papers}
\label{summary:auxiliarypapers}

In addition to the main publications listed in the previous section, the PhD
work done in the project has also contributed to other publications.
\lipsum[6-7]

\begin{description}
    % digital finishing
\item[\cref{pap:auxpaper1}] Leendert Wilhelmus Marinus Wienhofen and
  Bjørn Magnus Mathisen. \enquote{Defining the initial case-base for a
    CBR operator support system in digital finishing}. In: \emph{Goel, Ashok;
    D{\'i}az-Agudo, M Bel{\'e}n; Roth-Berghofer, Thomas (Ed.): Case-Based Reasoning Research and Development - 24th
    International Conference, ICCBR 2016}, Atlanta, GA, USA, October 31 - November 2,
    Proceedings, pp. 430–444, Springer, 2016.\\
    In this paper, we described the initial design and prototype implementation of a
    CBR-based operator support system.
  \item[\cref{pap:auxpaper2}] Kerstin Bach and Bjørn Magnus Mathisen and Amal Jaiswal.\textbf{Demonstrating the \protect\textsc{myCBR} Rest
      API}. In \emph{Kapetanakis, Stelios; Borck, Hayley (Ed.): Workshops
      Proceedings for the Twenty-seventh International Conference on Case-Based
      Reasoning co-located with the Twenty-seventh International Conference on
      Case-Based Reasoning (ICCBR 2019)}, Otzenhausen, Germany, September 8-12,
    2019, pp. 144–155, CEUR-WS.org, 2019. \\
    A paper detailing the expansion of myCBR with a REST interface and demonstrating how this increases usefulness
    for practitioners, researchers, CBR students, and teachers.
  \end{description}

\section{Source code}
\label{sec:othercontrib}
In addition to scientific progress and publications, the PhD work has produced
source code..
\lipsum[6-7]


\label{sec:othercontrib:sourcecode}
\begin{itemize}
\item Extended siamese neural network and surrounding test suite, implemented in
  Keras 2 and tensorflow 2 - Used in paper \cref{pap:paper1}\newline
  \url{https://github.com/ntnu-ai-lab/esnn}
\item New version of Extended siamese neural network and surrounding test suite
  - Used in paper \cref{pap:paper2}.\newline
  \url{https://github.com/ntnu-ai-lab/esnn-aqcbr}
\item As part of the work paper \cref{pap:auxpaper1} - Expansion of features in the
  myCBR CBR implementation \url{https://github.com/ntnu-ai-lab/mycbr-sdk} and
  co-creation of the REST API of myCBR
  \url{https://github.com/ntnu-ai-lab/mycbr-rest}
\item As part of the work done to replicate the similarity learning method from 
  in paper \cref{pap:auxpaper2} \url{https://github.com/ntnu-ai-lab/RProp}
\end{itemize}

%%% Local Variables:
%%% mode: latex
%%% TeX-master: "../main"
%%% TeX-engine: xetex
%%% End:

