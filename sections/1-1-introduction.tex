\chapter{Introduction}
\label{chap:intro}

\section{Motivation}
\label{sec:motivation}
\lipsum[1-2]
\section{Challenges}
\label{intro:sec:intro:challenges}
\lipsum[1-2]

\section{Research goals and research questions}
\label{intro:sec:rquestions}
\begin{tcolorbox}[title=Research goal]
  maybe a overarching research goal to answer the challenges here.
\end{tcolorbox}

Then we typically divide this research goal into research questions.

\begin{tcolorbox}[title=Research question 1]
  \label{rq1}
  RQ1 desc.
\end{tcolorbox}

\begin{tcolorbox}[title=Research question 2]
  \label{rq2}
  RQ2 desc.
\end{tcolorbox}

3...4 etc..

\section{Research context}
\label{sec:researchcontext}
Here you can describe the context of your phd project, what project it was part of - who gave you the data etc..
Maybe also what phases your phd project can be divided into..
\begin{itemize}
\item \textbf{Phase one}: Stumbling around and looking for ideas and data
\item \textbf{Phase two}: Here I we some data relating to the problem, and got a great idea - and made a new method. or something
\item \textbf{Phase three}: After i tested this method generally I applied it to the application domain problem because I finally got really relevant data from the problem owners.
\end{itemize}

\section{Thesis structure}
\label{sec:thesisstruct}

Describe the structure of the thesis.

This thesis is composed of two parts. Part one provides the overall motivation, structure, and main results from the thesis work and is divided into five
chapters. Chapter this presents that, chapter that describes the background.. Part two contains the four main papers published as part of the thesis work and
three auxiliary papers.


%%% Local Variables:
%%% mode: latex
%%% TeX-master: "../main"
%%% TeX-engine: xetex
%%% End:
